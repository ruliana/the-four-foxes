% Intended LaTeX compiler: pdflatex
\documentclass[11pt]{article}
\usepackage[utf8]{inputenc}
\usepackage[T1]{fontenc}
\usepackage{graphicx}
\usepackage{grffile}
\usepackage{longtable}
\usepackage{wrapfig}
\usepackage{rotating}
\usepackage[normalem]{ulem}
\usepackage{amsmath}
\usepackage{textcomp}
\usepackage{amssymb}
\usepackage{capt-of}
\usepackage{hyperref}
\usepackage{csquotes}
\title{The Four Foxes}
\hypersetup{
 pdfauthor={Ronie Uliana},
 pdftitle={The Four Foxes},
 pdfkeywords={},
 pdfsubject={},
 pdfcreator={Overleaf}, 
 pdflang={English}}
\begin{document}

\tableofcontents

\maketitle

\section{The Four Foxes}
\label{sec:intro}
\dots is a private detective agency that ran from 1923 through 1951.

The agency had some sort of "supernatural magnetism" that made every case an extraordinary one.

Four brothers, sons of a renowned explorer, ran the agency during its whole existence.

Differently from other games, where the focus is the scenario, here the focus is the four main characters. They were designed to complement each other and create some funny interactions between them.

\section{The characters in 1926}
\label{sec:characters}

\subsection{Moe Fox - \enquote{Grumpy Moe}}
\label{sec:moe}
Moe is in his early 50s. He was never and action guy but always felt responsible for his brothers since their father disappeared.

Motto: "Nobody hit my brothers, except me!"

\subsubsection{Can and Can't}
\label{sec:moe-can}
\begin{itemize}
\item Cannot win a close fight, unless one of his brothers is in danger. If so, he cannot lose a fight as far as his opponent is human.
\item Can clear fear, dizziness or any temporary mental condition from his brothers with a slap in the face or a punch.
\item Immune to fear when smoking a cigar.
\item Can sense danger or supernatural forces in a place or object. He can't say exactly what it is, but knows where and how bad it is.
\item Can cast a small flame on his hands, as strong as a lighter.
\item If shooting or throwing, he has the same precision of a drunk raccoon.
\end{itemize}

\subsection{Larry Fox - "Bold Larry" ("Reckless Larry" according to his brothers)}
\label{sec:orgcf82d2d}
Larry is in his middle 40s. He's a natural explorer as his father, and besides his age, is the most physically capable of the brothers.

Motto: "Me first!"

\subsubsection{Can and Can't}
\label{sec:org9c9803a}
\begin{itemize}
\item Can answer any question about ancient history or artifacts, or know where to find the answer given enough time.
\item Can fight 2 or 3 humans bare hands and win.
\item Never miss a shot or a throw, as far he has enough time to aim.
\item Can swing on ropes, jump over huge gaps, climb over huge walls. No obstacle can block him unless it's a dead end.
\item Cannot convince, coerce, or intimidate anyone.
\item Cannot lie convincingly, no matter the consequences.
\end{itemize}

\subsection{Sam Fox - "Cunning Sam"}
\label{sec:orgc38bb3d}
Sam is in his late 30s. He's the only real detective of the four brothers.

Motto: "It's obvious, isn't it?"

\subsubsection{Can and Can't}
\label{sec:orge2ce8a5}
\begin{itemize}
\item If sober and given enough time, can find anything odd on a place or situation. It's always something relevant to the story.
\item If sober and given enough time, can tell exactly what recently happened on a place.
\item When sober, can tell if someone is lying and his or her feelings during a conversation.
\item Can defeat any human opponent on a close combat, as far he's drunk or if he can improvise a weapon. No matter how big or small is the improvised weapon, but it's gonna break after the first use.
\item Can jump, climb and escape any pursuer when drunk. Can do the same when sober, but usually he got badly injured in the end.
\end{itemize}

\subsection{Lester Fox - "Smooth Lester"}
\label{sec:orgbd708ab}
Lester is in his middle 20s. The youngest brother is as \emph{bon-vivant} as he's good with people.

Motto: "Relax, ok? I have it under control."

\subsubsection{Can and Can't}
\label{sec:org2925e14}
\begin{itemize}
\item Given time, can convince anyone to do whatever he wants, as far it's not obviously life-risking.
\item Can lie as convincingly as telling the true. Maybe more convincingly than the true.
\item Cannot win any fight, no matter how. But can talk enough to make the opponent hesitate for a moment, even if it's not human.
\item When in a life-threatening situation, can run as faster as an olympic champion.
\item Given enough time, can disguise himself as anyone he had time to study.
\item Can sense any danger if he's behind someone else.
\end{itemize}

\section{Rules}
\label{sec:org134220e}
\subsection{Can and can't}
\label{sec:org2594811}
You don't have attributes or dices. Everything different from an "average human" is in that list of "can and can't," as also as your limitations.

"Can and can't" are absolute, if you "can," you can. Nothing will prevent you from doing that. The same for the "can't," if you can't do something, no matter what, you can't. The rules seem strict, but that what makes the game fun.

As we have no dices, no danger comes instantly and kills your character suddenly. The Game Master will always give you the picture, your options and tell if you are going well or not (more in the last section). So, please pay close attention to the clues the Game Master is telling you about how difficult it is.

\subsection{Fear, insanity, and snapping}
\label{sec:org6f20fb4}
Fear happens when:
\begin{itemize}
\item You see humans dying.
\item You are being hunted.
\item You see the supernatural.
\item You are risking your life willing.
\end{itemize}

Each one of those events make you lose one of you "can and can't"s, no matter if good or bad, until you have a good night of sleep on a safe place. So the effect is that the very beginning of fear makes you strong, as you're probably crossing your bad traits first. However, soon the fear will make you lose your good parts.

When you cross the \textbf{third} "can and can't", you need to choose one of you abilities that's not yet conditioned (meaning it has an "if" or "when") and create a condition.

After you create the condition, you clean all your fear, recovering all yours "can and can't".

The condition can be:
\begin{itemize}
\item An object you can lose. Like pendants, books, guns, hats or a particular piece of clothing. If you choose a ring or something tied to you, it'll be more difficult to lose it, but when you do, you'll lose you finger.
\item An indulging activity. That includes smoking, drinking, sex, praying, or anything that you can skip, for one reason or another. Like run out of cigars or booze, have no human around for sex or have to be in complete silence while hiding.
\item A mannerism. Snap your fingers, the sign of the cross, kiss a ring, or anything that you, as a player, could forgot to say your character is doing before the action.
\end{itemize}

If all your "can and can't" already have a condition, you "snap". That means you see the world a bit different, so you can do something extremely well.

When you "snap", add one more "can and can't". Can be something slightly supernatural or just a little above the better human limit.

Optionally, you can choose a "can't" and clean one of the conditions.

Keep the "can and can't" aligned with the characters profile and motto.

\section{Diceless - for Players}

Maybe this will be your first game without using dice. If you are an experienced player, you might be wondering if your GM wouldn't have too much control over your character. Actually, that's exactly the opposite as you know exactly what your character \textit{can} do without failing and what he or she \textit{can't} do, no matter what. The tricky part lies in between these two extremes.

So, here are only the two things you need to know, besides your character personality and the "can/can't" list.

\textbf{Ask what your character thinks.} You, as a player, don't have your character abilities. If he or she is an experienced brawler, no one expects you to evaluate an opponent by yourself just based on the Game Master's description. Just ask "do I think we can handle him?" and the Game Master will give you an accurate picture of what to expect and maybe some options. Your character is also an investigator, which very few of us are in real life, so a common question is "what do I think is worth noting here?", to which the Game Master always will point you to the relevant direction.

\textbf{Pay attention to damage.} That's how the GM tells you that the course of actions is not working. Your character is never going to die from a sudden explosion without warning, but more likely die fighting a stronger opponent alone, ignoring all the increasing damage he's inflicting upon you. So, when the Game Master tells you "the rocks are still hitting you, and they're getting bigger", that means you need to do something different, and do it fast. Some situations are really dangerous and there is no way to get out of it without some damage, it's a matter of choosing the "lesser evil". And that's a big part in an horror/investigative game.


\section{Diceless - for Game Masters}
\label{sec:org2563cf9}
A diceless game is a challenge one. The most important thing is to let your players clearly knows what's the consequences of each action.

So, you guidelines are:
\begin{itemize}
\item Always answer "yes, you can". The characters are the heroes and are above normal people, so the default answer is always "yes" and the action is always a success.
\item Sometimes answer "yes, you can. But\ldots{}". The characters have "cant's" on their list, so if the player attempts something his character can't do, tell how that is going to fail. The consequences are never terminal, this is not a game characters die unless their players want to do so. Dead is always intentional. Fear and insanity, on the other hand, are not.
\end{itemize}

The world doesn't stop as the players talk. No matter if they are talking in or out character. Every minute or so you must interrupt them and tell that the world is moving. That can be their pursuers getting closer; or they hear, smell or sense something interesting; maybe someone enters the room and drops dead. Unless they are in a really safe place, the world is not going to wait for them. In the moments of high tension, don't let them coordinate their actions, after one or two phrases between them, tell the cultists are getting closer, then dangerously closer, and finally that they engage in combat.

From time to time, the players will be in fierce combat or in a place you want to picture as dangerous. The most important rule here is: \textit{any pause to debate means damage}. There is no time to talk, that's why those situations are dangerous (and exciting).

\subsection{Examples}
\label{sec:orgcef5c31}
The brothers are fleeing from some cultists and they got cornered on a cliff.

GM - "You are being hunted by those cultists. One fear for everyone and cancel a "can and can't" until you are in a safe place.

Moe - "We should jump, we are not going to die."

GM - "Probably not, but you can break a leg or an arm, and surely that will ruin your cigars."

Moe - "Hmmm\ldots{} I don't like that. Hey, guys. I can fight if you do."

Lester - "I'd rather run. Can I disguise as one of the cultists?"

GM - "Yes, but you need time you don't have. Unless you think something you can do to buy you time. By the way, the cultists are getting closer, you can see them coming. You don't have much time."

Larry - "How many of them? I can take some by myself."

GM - "I would say 10 or 12, it's difficult to see in the dark. You think it's a bit too much for you alone."

Larry - "Ok. Guys, let's go fight?"

Moe - "I'm down!"

Lester - "That can create a distraction, so I can disguise myself."

GM - "Indeed."

Sam - "I don't know, I'm out of booze. I'd rather try the cliff."

Moe - "Bad idea, and also they can wait for us at the beach."

Larry - "Yeah. I think you should stay behind us."

Sam - "Wait, I can improvise a weapon. Any rock will do!"

GM - "Yes, that will work. And now the cultists are really close. They are going to reach you in seconds, what are you doing?"

Lester - "Wait, I can cancel my 'cannot win a fight' due to fear."

GM - "Yes, you already chose something else, but no problem, let's assume you have chosen differently."

Lester - "Nice! So, what if \ldots{}"

GM - "The cultists are upon you, and they start to fight. They have no weapons, but they are a stronger than you expect."

\subsection{Combat}
\label{sec:orge74e7bb}
Diceless combat is more strategic than what you are use to. Instead of describing each blow, the Game Master will ask about what the player wants to achieve and how the character is fighting. Each "step" is a series of blows that can be seconds to minutes.

\textit{Bruises and damage are clues to the players that the actions are going in the wrong direction}. The worst is the damage, less time players have to change their strategy. However, don't break an arm without a clear warning that can happen. Increase the injuries as the fight goes until you reach the point for some nasty damage.

\subsubsection{Example}
\label{sec:org0a9b144}

Sam - "I'm going to hit the closest cultist with my rock."

GM - "Ok, the rock is not going to last long, but it'll do for a while. What do you want to do? Kill them? Just knock them out? Hold them to give time to Lester?"

Sam - "Do I think we can defeat them?"

GM - "You have seen a lot of brawls and you're pretty sure you can beat them, but not without a lot of fatigue and some broken bones."

Sam - "Ouch! Lester, can you do something?"

Lester - "I guess I can trick them if they think I'm one of them."

GM - "That works."

Sam - "Ok, I'm just holding my ground, but making a lot of trouble to call their attention."

Larry - "I'm with Sam."

Moe - "I'm going to lit my cigar to cancel the fear and join the fray."

GM - "You're in a hurry, but you got your cigar lit. Are you all buying time for Lester?"

Larry - "Yes."

GM - "Ok, you are more interested on keeping them looking at you than really knock them out. That works really well, however, they are slowly pushing you towards the cliff. Lester, your move."

Lester - "I'm going to reach the cultist who is behind the others and hit him hard."

GM - "What do you want to do? Kill him?"

Lester - "Nah, I want his robes."

GM - "No big deal. Everyone is paying attention to you brothers, so it's easy to sneak behind one of them. You have you robes and a guy laying on the ground."

Lester - "I'm going to kick him over the cliff."

GM - "Done. He's passed out, nobody really noticed you. However, guys, you are dangerously close to that cliff."

Moe - "Kid, whatever is your plan, it's better to do it now." - to the GM - "I'm going to press them a little harder."

GM - "A little dangerous, but you got some ground at cost of some bruises."

Lester - "I'm running away the cliff and \ldots{}"

Moe - "I'm gonna kill you, you coward!"

Lester - "Let me finish! And I'm going to shout to the cultists - 'Brothers! Forget those intruders, the Beast will take care of them, let's return before it reaches us too!"

GM - "Ahhh\ldots{} good call. You saw the Beast on a cage before. The cultists hesitate for a moment."

Lester - "You know the Beast will kill us too if it got us here!" - to GM - "I'm bluffing, I had no idea what that Beast does."

GM - "It's a fair assumption. The cultists start to retreat, you can sense they are on the verge of panic."

Lester - "Nice!"

Moe - "Good job, kid!"

Lester - "Don't call me that!"

GM - "You hear a howl at distance. Pretty sure it's not a dog. What are you doing?"

\section{A night in the ship}
\label{sec:one-shot-a-night-in-the-ship}

Lester got some tickets for a one week cruise from "a friend of a friend". It would be an excellent opportunity to relax after the stress of the last case.

You have embarked aboard the "Rising Star" in the late afternoon, and everything was running smoothly as silk. The ship had a casino, a theatre, several bars and games on the main deck.

Early in the next morning, the Captain summons you to the casino. When you arrived, you see the doors close, a small sign "under maintenance, sorry for the inconvenience", and two big guys from the crew by each side. They introduce you to the big hall. The lights were dimmed, and the place was utterly silent, except by the voice of two men talking in the back.

When you arrived there, the Captain and their Safety Officer were talking. The Captain saluted you gravely and led you to a small room in the back, where a man from the crew was laying on the ground, beheaded. A pentagram was drawn on the floor, close to the body. His head, without the eyes, was carefully aligned in one of the vertices.

The Captain says: "We have a murderer here, and I think there will be more victims. I want to hire your services and your discretion". 

The game starts now.

\end{document}
